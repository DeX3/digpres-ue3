\section{Dataset Description}\label{sec:dataset-description}

This section explains the created dataset.

\subsection{Export}\label{sec:export}

Because we used MongoDB as our database, it is straightforward to select JSON as the
format for our export. It is a nice, human-readable format that can be easily
exported by tools such as \texttt{mongoexport}:

\begin{lstlisting}
$ mongoexport --db github \
              --collection github-1433519125507 \
              --jsonArray \
              --out export.json
connected to: 127.0.0.1
exported 1000 records
\end{lstlisting}

\subsection{Dublin Core}\label{sec:dublin}

The idea of Dublin Core is to standardize the metadata for physical resources
and web resources. Appendix \ref{sec:dc} shows an example for the Dublin Core
Scheme for our solution. In our opinion it is a great scheme to compare datasets
and get information but it is not really applicable for every dataset. Our
dataset is one example, where we think it is not really applicable and where the
scheme can not show its strength, because the metadata for every dataset in
the future is most likely the same except the date. 

\subsection{Preservation Metadata: Implementation Strategies (PREMIS)}\label{sec:premis}
This section describes the PREMIS metadata format.

\subsubsection{Most relevant metadata properties for our
dataset}\label{sec:propertylist}

\paragraph{title}
The title of the dataset.
\paragraph{description}
The description of the dataset.
\paragraph{subject}
The subject of the dataset.
\paragraph{date created}
The date when the dataset was created.
\paragraph{creator}
The person who created the dataset.
\paragraph{publisher}
The publihser of the dataset.
\paragraph{format}
The format of the dataset.
\paragraph{size of file}
The size of the dataset.
\paragraph{number of repositories}
The number of repositories which were evaluated in this dataset.
\paragraph{rightsholder}
The rightsholder of the dataset.
\paragraph{repo-hosting service}
The repositories-hosting service of the dataset.
\paragraph{workflow version}
The workflow-version of the dataset.
\paragraph{identifier}
The identifier of the dataset.
\paragraph{language}
The language of the dataset.
\paragraph{provenance}
The provenance of the dataset.

\subsubsection{Description and dissusion about metadata
properties}\label{sec:descprop}

We defined our metadata-properties in section \ref{sec:propertylist}. We added
some metadata which are specific for our solution such as number of repositories
and workflow version to make the files better comparable and get more date out
of them. We also made a metadata-property repo-hosting service, because it is
possible that we have to change from GitHub to another repo-hosting service in
he future.


\subsubsection{Table of our metadata records}\label{sec:metadatarecordtable}

\begin{tabular}{|c|l|}
    \hline
    \textbf{Metadata-property} & \textbf{Metadata-value} \\
    \hline
    title & GitHub Repositories \\
    \hline
    description & 1st 1000 repos returned by https://api.github.com/repositories  \\
    \hline
    subject & 1st 1000 GitHub repos \\
    \hline
    date created & 2015-06-06 \\
    \hline
     creator & Roman Decker \\
    \hline
     publisher & GitHub \\
    \hline
     format & JSON \\
    \hline
     size of file & 206.773 Byte \\
    \hline
     number of repositories & 1000 \\
    \hline
     rightsholder & GitHub \\
    \hline
     repo-hosting service & github.com \\
    \hline
     workflow version & 1 \\
    \hline
     identifier & dataset \\
    \hline
     language & english \\
    \hline
     provenace & GitHub \\
    \hline
\end{tabular}
